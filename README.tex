\documentclass[11pt]{article}
\usepackage{xcolor}
\definecolor{darkblue}{rgb}{0,0,0.33}
\usepackage{hyperref}
\newcommand{\link}[2]{\underline{\color{darkblue}\href{#1}{#2}}}
\renewcommand{\familydefault}{\sfdefault}
\usepackage{parskip}

\begin{document}
\title{\textbf{fountain-parser v.0.1.0.0 README}}
\author{}
\date{}
\maketitle

\section*{Synopsis}
\texttt{fountain-parser} is a small parser library for the
\link{https://fountain.io/}{\textrm{Fountain}} screenplay format,
fully supporting 1.1 version
\link{https://fountain.io/syntax/}{syntax} and
producing a simple, easy to grok \textrm{AST}.

\texttt{fountain-parser} is written in
\link{https://haskell.org}{\textrm{Haskell}} and it uses the
\link{https://hackage.haskell.org/package/megaparsec}{\textrm{Megaparsec}}
library for parsing. Megaparsec might sound like overkill for this
modest purpose, but 

\section*{Disclaimer}
Currently, this is \emph{pre-alpha} software, not yet usable in
productive form.

This software is distributed under the
\emph{\textrm{BSD Three-Clause license}}.
See the \link{run:./LICENSE}{LICENSE} file for more details.

\section*{Motivation}
The ``\emph{Developers}'' section of the Fountain site provides a
link to a
\link{https://github.com/nyousefi/Fountain}{parsing library}
in \textrm{Objective~C}. This already presents a portability
issue: there \emph{are} projects that make it possible to bridge
Objective C and Haskell, but they're platform- or
framework-specific.  It also employs a multi-pass stategy where
every stage creates a modified version of the source, and it's
heavily reliant on \emph{Regular Expressions}.

Thus, to create a light-weight, performant and portable
solution, it's necessary to start from scratch.

\texttt{fountain-parser} aims to power a series of command-line
utilities for conversion from Fountain to a series of convenient
formats, (\texttt{.OTF}, \texttt{.TEX}) without intervention
from thirds.

  \subsection*{My software already supports Fountain}
  Of course. The
  \link{https://fountain.io/apps/}{``\emph{Apps}''
  section} of the Fountain site lists a few that also import or
  export the format. \textbf{The caveat}: most are either
  cloud-based and/or proprietary. By favoring (mostly) open
  formats, \emph{fountain-parse} allows integration into many
  \textrm{FLOSS} tools, helping the creation of compound
  documents (such as production bibles) and entirely
  non-proprietary workflows.

\section*{Implementation Specifics}
  \begin{itemize}
    \item As per spec:
      \begin{itemize}
        \item This library expects Fountain text to be
          encoded in \textrm{UTF-8}.
        \item Tabs are converted into \textbf{four} spaces.
        \item Your line-spacing is respected.
        \item Initial spaces are ignored everywhere except in
          non-\texttt{\textgreater{}centered{}\textless} action
          lines.
        \item A line with two spaces doesn't count as
          an empty line.
      \end{itemize}
    \item All parsing functions expect \texttt{Text} inputs.
      File I/O is left to the application or framework.
    \item Formatting (boldface, underline) found in such
      entities as character names or scene headings is
      ignored.
    \item Vertical tabs and form-feed characters are
      interpreted as line changes. For vertical spacing, use
      multiple blank lines and/or the Fountain form feed
      character sequence (``\texttt{===}'') instead.
    \item Unicode spaces are turned into the vanilla space,
      except for the \emph{hair space}, which is discarded.
    \item The parser keeps everything: notes, boneyards,
      sections and synopses. Some possible conversion targets
      have equivalents and might want to conserve them.
  \end{itemize}

  \subsection*{Tentative Grammar}
  While the \textrm{Fountain} spec does not have a
  

\section*{Building}
\textrm{GHC} 9.6.7 and \textrm{Cabal} 3.0 (or greater) are
required to compile and run the test suite (once implemented.)

The project uses the \texttt{GHC2021} language default. While
it might be possible to compile it in earlier versions than
9.6.7, this default is only available since 9.2.1., so that
constitutes a hard version limit for those who might wish to
experiment.

Some of the included scripts require Linux or a Linux-like
environment (e.g.,
\link{https://www.msys2.org/}{\textrm{MSYS2}}.)

\section*{Contact}
Please
\link{https://github.com/CubOfJudahsLion/fountain-parser/issues}{create an issue}
if you find a bug.

I can be reached directly at
\emph{\textrm{10951848+C\"{u}b\b{O}fJ\'{u}d\~{a}hsL\^{i}\`{o}n}}
\u{a}(t) \emph{\textrm{users/noreply/g\={i}th\d{u}b/c\.{o}m}}
(without accents and replacing slashes by periods.)

\end{document}
