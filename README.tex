\documentclass[12pt]{article}
\usepackage{xcolor}
\definecolor{darkblue}{rgb}{0,0,0.33}
\usepackage{hyperref}
\newcommand{\link}[2]{\underline{\color{darkblue}\href{#1}{#2}}}
\usepackage{parskip}

\begin{document}
\title{\textbf{fountain-parser README}}
\author{}
\date{}
\maketitle

\texttt{fountain-parser} is a small parser library for the
\link{https://fountain.io/}{Fountain} screenplay format,
supporting 1.1 version
\link{https://fountain.io/syntax/}{syntax} and
producing a simple, easy to grok \textsf{AST}.

\texttt{fountain-parser} is written in
\link{https://haskell.org}{Haskell} and it uses the
\link{https://hackage.haskell.org/package/megaparsec}{Megaparsec}
library for parsing.

\section{Motivation}
The ``\emph{Developers}'' section of the Fountain site provides a
link to a
\link{https://github.com/nyousefi/Fountain}{parsing library}
in \textsf{Objective~C}. This already presents a portability
issue:there \emph{are} projects that make it possible to connect
Objective C to Haskell, but they're either platform- or
framework-specific.  It also employs a multi-pass stategy where
every stage creates a modified version of the source, and it's
heavily reliant on \emph{Regular Expressions}.

Thus, to create a light-weight, performant and portable
solution, it's necessary to start from scratch.

\texttt{fountain-parser} aims to power a series of command-line
utilities for conversion from Fountain to a series of convenient
formats, such as \texttt{.OTF} or \texttt{.PDF}, without the
intervention of thirds.

\section{Implementation Specifics}
Fountain files are \textsf{UTF-8} text files. While this library
doesn't impose any file naming scheme, it is customary for Fountain
files to have the extension \texttt{.fountain}, \texttt{.txt} or
\texttt{.spmd} (meaning
``\textbf{S}creen\textbf{p}lay \textbf{M}ark\textbf{d}own'',
one of the formats that eventually merged into Fountain.)




\end{document}
