\documentclass[12pt]{article}
\usepackage{xcolor}
\definecolor{darkblue}{rgb}{0,0,0.33}
\usepackage{hyperref}
\newcommand{\link}[2]{\underline{\color{darkblue}\href{#1}{#2}}}
\usepackage{parskip}

\begin{document}
\title{\textbf{fountain-parser README}}
\author{}
\date{}
\maketitle

\section*{Synopsis}
\texttt{fountain-parser} is a small parser library for the
\link{https://fountain.io/}{\textsf{Fountain}} screenplay format,
fully supporting 1.1 version
\link{https://fountain.io/syntax/}{syntax} and
producing a simple, easy to grok \textsf{AST}.

\texttt{fountain-parser} is written in
\link{https://haskell.org}{\textsf{Haskell}} and it uses the
\link{https://hackage.haskell.org/package/megaparsec}{\textsf{Megaparsec}}
library for parsing.

\section*{Status}
Currently, this is \emph{pre-alpha} software, not yet usable in
any form. We'll have something testable soon enough.

\section*{Motivation}
The ``\emph{Developers}'' section of the Fountain site provides a
link to a
\link{https://github.com/nyousefi/Fountain}{parsing library}
in \textsf{Objective~C}. This already presents a portability
issue: there \emph{are} projects that make it possible to connect
Objective C to Haskell, but they're either platform- or
framework-specific.  It also employs a multi-pass stategy where
every stage creates a modified version of the source, and it's
heavily reliant on \emph{Regular Expressions}.

Thus, to create a light-weight, performant and portable
solution, it's necessary to start from scratch.

\texttt{fountain-parser} aims to power a series of command-line
utilities for conversion from Fountain to a series of convenient
formats, such as \texttt{.OTF}, \texttt{.TEX} or \texttt{.PDF},
without the intervention of thirds.

\section*{Implementation Specifics}
Fountain files are \textsf{UTF-8} text files. While this library
doesn't impose any file naming scheme, it is customary for Fountain
files to have the extension \texttt{.fountain}, \texttt{.txt} or
\texttt{.spmd} (meaning
``\textbf{S}creen\textbf{p}lay \textbf{M}ark\textbf{d}own'',
the format that eventually became Fountain.)

\subsection*{But my software already supports Fountain!}
The \link{https://fountain.io/apps/}{``\emph{Apps}''
section} of the Fountain site lists a number of apps that can
import or export the format. The caveat: most are either
cloud-based and/or proprietary. By favoring (mostly) open formats,
\emph{fountain-parse} allows integration into many FLOSS tools,
helping the creation of compound documents (such as production
bibles) and entirely non-proprietary workflows.

\section*{Contact}
Please create an issue if you find one.

I can be reached directly at
\emph{\textsf{10951848+C\"{u}b\^{O}fJ\'{u}d\~{a}hsL\^{i}\`{o}n}
at \textsf{users/noreply/g\^{i}th\~{u}b/c\"{o}m}} (without accents and
replacing slashes by periods.)

\end{document}
